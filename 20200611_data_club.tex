%% Ankur Sinha

%% packages %%
% support for coloured text
\usepackage{color}
% IPA
\usepackage{tipa}
\usepackage[scale=2]{ccicons}
\usepackage{amssymb}
\usepackage{tikz}
\usetikzlibrary{mindmap, arrows.meta, positioning, arrows}
\usepackage{pgfplots}
% Define the colours we use for E and I in all graphs
\definecolor{SinhaBlueE}{HTML}{3b4cc0}
\definecolor{SinhaRedI}{HTML}{f7a789}
\pgfmathdeclarefunction{gaussnew}{4}{%nu, eta, eps, omega
  \pgfmathparse{(#1*((2*exp(-(((x-((#2+#3)/2))/((#2-#3)/(2*sqrt(-ln(#4/2)))))^2))) -#4))}%chktex 36
}
\usepackage{jneurosci}
\usepackage{subcaption}
\usepackage[T1]{fontenc}
\usepackage[utf8]{inputenc}
\usepackage[style=nature,backend=biber,autocite=footnote]{biblatex}
% \addbibresource{/home/asinha/Documents/01_Readables/00_research_papers/bibliography/masterbib.bib}
% \renewcommand*{\bibfont}{\tiny}
% Use opensans
% \usepackage[default,scale=0.95]{opensans}
\usepackage[sfdefault]{roboto}
% for strike through
\usepackage[normalem]{ulem}
% links, urls, refs
\definecolor{links}{HTML}{2A1B81}
% Fedora blue for the theme
\definecolor{FedoraBlue}{HTML}{2A1B81}
\usepackage{hyperref}
\hypersetup{colorlinks,linkcolor=Green,urlcolor=links}
% graphics
\usepackage{graphicx}
% algorithm
\usepackage{algorithmic}
\usepackage{textcomp}
\usepackage{wrapfig}
\usepackage{textgreek}
\usepackage{euler}
\usepackage{csquotes}
\usepackage{tabularx}
\usepackage{booktabs}
% beamer theme
% use defaults for theme
\usetheme[numbering=fraction]{metropolis}
\usefonttheme[onlymath]{serif}
\setbeamerfont{footnote}{size=\tiny}
\setbeamerfont{caption}{size=\tiny}
\setbeamercolor{alerted text}{fg=Green}
\setbeamerfont{note page}{size=\small}

% Not needed in metropolis, but in general footnote citation fixes: https://tex.stackexchange.com/questions/44217/how-can-i-stop-footcite-from-hijacking-my-beamer-columns
% how to use multiple references to the same footnote: https://tex.stackexchange.com/questions/27763/beamer-multiple-references-to-the-same-footnote

% Disable footnoterule
\renewcommand{\footnoterule}{}

%% title %%
\title{Data club: OSB updates}
\subtitle{\enquote{Show and tell}}
\author[Ankur Sinha]{Ankur Sinha}
\date{11/06/2020}

%% document begins %%
\begin{document}


% title frame %%
\begin{frame}
  \titlepage{}
\end{frame}
%% Three slides for 5 minutes seems good
%% So, 30 slides at most for 50 minutes
\begin{frame}[c]
  \frametitle{OSBv1}

  Demo: \url{www.opensourcebrain.org}
  \note{Mostly showing various bits, not too many slides needed.}

\end{frame}
\begin{frame}[c]
  \frametitle{OSBv1: current status}
  \begin{itemize}
    \item Maintenance mode: small bug fixes, minor enhancements:
      \begin{itemize}
        \item \href{app.zenhub.com}{Ankur's Kanban board (public)}.
      \end{itemize}
      \pause{}
    \item Consolidation of infrastructure on Google Cloud Services:
      \begin{itemize}
        \item kubernetes (k8s):
          \begin{itemize}
            \item also container based but,
            \item better fine grained control of computing resources.
          \end{itemize}
        \item Continuous Deployment (CD) pipeline:
          \begin{itemize}
            \item GitHub \(\rightarrow{}\) CodeFresh \(\rightarrow{}\) Google cloud \(\rightarrow{}\) live!
          \end{itemize}
          \pause{}
        \item MetaCell:
          \begin{itemize}
            \item CloudHarness: cloud infrastructure manager.
          \end{itemize}
        \item Test OSBv1 deployment: \url{beta.opensourcebrain.org}.
      \end{itemize}
  \end{itemize}
\end{frame}
\begin{frame}[c]
  \frametitle{OSBv1: future plans}
  Maintain until OSBv2 reaches feature parity.
\end{frame}
\begin{frame}[c]
  \frametitle{OSBv2: the idea: closing the loop}
    \begin{itemize}
      \item Modelling:
        \begin{itemize}
          \item NetPyNE:\ Simulation and analysis.
        \end{itemize}
      \item Experiments:
        \begin{itemize}
          \item NWBExplorer.
        \end{itemize}
        \pause{}
      \item Workspaces:
        \begin{itemize}
          \item both in one place,
          \item sharing and collaboration.
        \end{itemize}
    \end{itemize}
\end{frame}
\begin{frame}[c]
  \frametitle{NWBExplorer revisited}
  \url{http://nwbexplorer.opensourcebrain.org/}
\end{frame}
\begin{frame}[c]
  \frametitle{NetPyNE UI}
  \url{http://34.68.113.100/}
\end{frame}
\begin{frame}[c]
  \frametitle{OSBv2: preview}
  \url{http://www.v2.opensourcebrain.org/}
\end{frame}
\begin{frame}[c]
  \frametitle{OSBv2: what to expect: design mockups}
  \url{https://www.figma.com} (evolving, not public)
\end{frame}
\begin{frame}[c]
  \frametitle{OSBv2: progress and plans}
  \begin{itemize}
    \item Short term (this year):
      \begin{itemize}
        \item \href{https://app.zenhub.com/workspaces/open-source-brain-5c4f23e44c15c80eadb6c30d/board}{MetaCell's Kanban Board (public)},
        \item Outreach: early adopters, tweaks as per use cases,
        \item OSBv1 sunset: when OSBv2 is ready enough,
      \end{itemize}
      \pause{}
    \item Long term (3--5 years: feature complete OSBv2):
      \begin{itemize}
        \item \href{https://airtable.com}{MetaCell: user stories (airtable.com, not public)}.
      \end{itemize}
      \pause{}
    \item Long long term:
      \begin{itemize}
        \item Maintenance: user support, bug fixes, security updates, infrastructure updates,
        \item No end to integrations with new data sources, repositories.
      \end{itemize}
  \end{itemize}
\end{frame}

\end{document}

